\section{Конструкторская часть}

В данном разделе будут рассмотрены возможные сценарии неправомерных действий и признаки, по которым данные неправомерные действия могут быть обнаружены, будет рассмотрен алгоритм обновления хеш-сумм для связи всех кусков.

\subsection{Сценарии неправомерных действий}

На рисунке \ref{fig:scenarios} изображены сценарии возможного неправомерного доступа. Здесь рассматриваются только действия с данными, без метаданных.

\begin{figure}[hbtp]
	\centering
	\includegraphics[width=\textwidth]{img/scenarios.pdf}
	\caption{Сценарии неправомерных действий.}
	\label{fig:scenarios}
\end{figure}

\subsection{Схемы алгоритмов работы метода}

На рисунках \ref{fig:mainalgo} и \ref{fig:mainalgo2} показана схема нового алгоритма изменения состояний кусков. Серым отмечены блоки, которые присутствовали в старом алгоритме. Во внутренней структуре состояния кусков меняются следующим образом:
\begin{itemize}
	\item [---] кусок, который переименовывается из временного в обычный, переходит из временного состояния в активное;
	\item [---] куски, которые покрываются переименовываемым куском, переводятся из активного состояния в истекшее и не используются при чтении.
\end{itemize}

На рисунке \ref{fig:recalcalgo} показана схема алгоритма расчета цепи хеш-сумм.

На рисунке \ref{fig:checkalgo} показана схема алгоритма проверки цепи хеш-сумм.


\begin{figure}[hbtp]
	\centering
	\includegraphics[scale=0.8]{img/mainalgo.pdf}
	\caption{Схема нового алгоритма изменения состояний кусков.}
	\label{fig:mainalgo}
\end{figure}
\begin{figure}[hbtp]
	\centering
	\includegraphics[scale=0.8]{img/mainalgo2.pdf}
	\caption{Схема нового алгоритма изменения состояний кусков. Продолжение.}
	\label{fig:mainalgo2}
\end{figure}

\begin{figure}[hbtp]
	\centering
	\includegraphics[scale=0.8]{img/recalcalgo.pdf}
	\caption{Схема алгоритма расчета цепи хеш-сумм.}
	\label{fig:recalcalgo}
\end{figure}

\begin{figure}[hbtp]
	\centering
	\includegraphics[scale=0.75]{img/checkalgo.pdf}
	\caption{Схема алгоритма проверки валидности цепи хеш-сумм.}
	\label{fig:checkalgo}
\end{figure}

\subsection{Выводы}

В данном разделе были рассмотрены возможные сценарии неправомерного доступа, а также построены схемы алгоритмов, необходимых для реализации разрабатываемого метода.

\pagebreak