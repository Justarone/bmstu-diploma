\section*{ОПРЕДЕЛЕНИЯ, ОБОЗНАЧЕНИЯ И СОКРАЩЕНИЯ}

Хеш-функция --- функция, осуществляющая преобразование массива входных данных произвольной длины в выходную битовую строку установленной длины, выполняемое определенным алгоритмом \cite{hash}.

Хеш-сумма (хеш-код) --- строка бит, являющаяся выходным результатом хеш-функции \cite{hashsum}.

Криптографическая стойкость --- способность криптографического алгоритма противостоять криптоанализу; стойким считается алгоритм, успешная атака на который требует от атакующего обладания недостижимым на практике объемом вычислительных ресурсов либо настолько значительных затрат времени на раскрытие, что к его моменту защищенная информация утратит свою актуальность \cite{crypto}.

СУБД --- система управления базами данных \cite{dbms}.

Подсистема хранения ---  компонент СУБД, управляющий механизмами хранения баз данных, или библиотека, подключаемая к программам и дающая им функции СУБД \cite{dbengine}\cite{mysqlengines}.

\pagebreak