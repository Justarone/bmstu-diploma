\section*{РЕФЕРАТ}

Расчетно-пояснительная записка \pageref{LastPage} с., \totalfigures\ рис., \totaltables\ табл., 42 ист., 1 прил.

В работе представлена разработка метода блочного хранения данных с возможностью доказательства неправомерного доступа.

Проведена классификация видов защиты информации. Рассмотрены существующие системы блочного хранения данных с защитой от неправомерного доступа, их ограничения и области применимости, отмечены связи между свойствами методов и видами защиты информации. Рассмотрен метод блочного хранения в СУБД ClickHouse в движке MergeTree, проанализированы потенциальные угрозы. На основе имеющегося метода разработан метод, обеспечивающий возможность доказательства неправомерного доступа. Был проанализирован и протестирован разработанный метод с шифрованием данных и без него.

КЛЮЧЕВЫЕ СЛОВА

\textit{защита информации, Blockchain, шифрование, блочное хранение данных, ClickHouse, MergeTree.}

\pagebreak