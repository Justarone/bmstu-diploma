\section{Исследовательская часть}

В данном разделе будет показаны действия, которые требуется сделать, чтобы обмануть систему без шифрования данных, а также будет показано, что система обнаруживает неправомерные действия с наличием шифрования.

\subsection{Обход механизма защиты при отсутствии шифрования данных}

Рассмотрение для 2 форматов хранения:
\begin{enumerate}
    \item компактное;
    \item широкое.
\end{enumerate}

\subsubsection{Компактное хранение}

Необходимо создать таблицу и поместить в нее данные. Эти действия выполняются запросами, представленными в листинге \ref{code:createtable1}.

\begin{lstlisting}[language=sql, label=code:createtable1, caption={Создание таблицы с компактным хранением.}]
CREATE TABLE cats
(
    `id` UInt64 CODEC(NONE),
    `name` String CODEC(NONE),
    `age` UInt64 CODEC(NONE)
)
ENGINE = MergeTree
ORDER BY id
SETTINGS min_bytes_for_wide_part = 1000000000, min_rows_for_wide_part = 1000000000, use_parts_chainer = 1

INSERT INTO cats VALUES (1, 'Murka', 11), (2, 'Elsa', 18), (3, 'Cleo', 5)
\end{lstlisting}

Как было показано в конструкторской части, при изменении какого-то блока данных требуется пересчитать 3 хеш-суммы куска и цепь хеш-сумм. Пусть требуется выставить столбцу значение $age = 4$, где $id = 3$. Данные на диске выглядят так, как показано в листинге \ref{code:raw}.

\pagebreak

\begin{lstlisting}[label=code:raw, caption={Данные записанного куска в сыром виде.}]
hexdump -C all_1_1_0/data.bin
39 32 97 f2 20 a1 0f 9e  d5 82 4c 8a 1c 64 a7 74  |92.. .....L..d.t|
02 21 00 00 00 18 00 00  00 01 00 00 00 00 00 00  |.!..............|
00 02 00 00 00 00 00 00  00 03 00 00 00 00 00 00  |................|
00 5c 43 37 3f 94 e6 65  eb 4d fb 60 68 3a 35 85  |.\C7?..e.M.`h:5.|
4d 02 19 00 00 00 10 00  00 00 05 4d 75 72 6b 61  |M..........Murka|
04 45 6c 73 61 04 43 6c  65 6f d8 cd aa 90 8a 0a  |.Elsa.Cleo......|
4a 0a 5f c3 97 cd bd e9  58 3d 02 21 00 00 00 18  |J._.....X=.!....|
00 00 00 0b 00 00 00 00  00 00 00 12 00 00 00 00  |................|
00 00 00 05 00 00 00 00  00 00 00                 |...........|
\end{lstlisting}

Если разобрать по байтам, получится:
\begin{enumerate}
    \item 16 байт --- хеш-сумма для сжатого блока (1 столбца);
    \item 9 байт --- заголовок кодека, состоящий из:
    \begin{itemize}
        \item [---] 1 байт --- метод;
        \item [---] 4 байта --- размер сжатых данных (с заголовком);
        \item [---] 4 байта --- размер исходных данных.
    \end{itemize}
    \item 24 байта --- 3 числа по 8 байт --- столбец \texttt{id};
    \item 16 и 9 байт аналогично на хеш-сумму и заголовок кодека для 2 столбца;
    \item 16 байт --- строки <<Murka>>, <<Elsa>> и <<Cleo>> и перед ними числа размером 1 байт;
    \item 16 и 9 байт аналогично на хеш-сумму и заголовок кодека для 3 столбца;
    \item 24 байта --- 3 числа по 8 байт --- столбец \texttt{age}.
    \end{enumerate}
\end{enumerate}

Таким образом требуется поменять:
\begin{enumerate}
    \item 1 байт --- значение числа, которое хотим поменять;
    \item 16 байт --- хеш-сумма столбца (блока сжатых данных).
\end{enumerate}

После замены файл \texttt{data.bin} будет выглядеть как в листинге \ref{code:newdata}.

\pagebreak

\begin{lstlisting}[label=code:newdata, caption={Данные записанного куска в сыром виде после исправления.}]
39 32 97 f2 20 a1 0f 9e  d5 82 4c 8a 1c 64 a7 74  |92.. .....L..d.t|
02 21 00 00 00 18 00 00  00 01 00 00 00 00 00 00  |.!..............|
00 02 00 00 00 00 00 00  00 03 00 00 00 00 00 00  |................|
00 5c 43 37 3f 94 e6 65  eb 4d fb 60 68 3a 35 85  |.\C7?..e.M.`h:5.|
4d 02 19 00 00 00 10 00  00 00 05 4d 75 72 6b 61  |M..........Murka|
04 45 6c 73 61 04 43 6c  65 6f 75 37 d2 0b 34 77  |.Elsa.Cleou7..4w|
72 89 c2 7a f3 1c 62 ef  ee 97 02 21 00 00 00 18  |r..z..b....!....|
00 00 00 0b 00 00 00 00  00 00 00 12 00 00 00 00  |................|
00 00 00 04 00 00 00 00  00 00 00                 |...........|
\end{lstlisting}

При проверке состояния таблицы получается результат, описанный в листинге \ref{code:errchecksum}.

\begin{lstlisting}[label=code:errchecksum, caption={Результат проверки.}]
CHECK TABLE cats
format Vertical

Query id: dbcbc244-10f0-491d-85e7-849377ac73e0

Row 1:
──────
part_path: all_1_1_0
is_passed: 0
message: Checksum mismatch for file data.bin in data part
\end{lstlisting}

Была заменена хеш-сумма сжатого блока, однако не были заменены хеш-суммы сжатых и расжатых данных, описанные в \texttt{checksums.txt}, который описан в листинге \ref{code:checksumsold}.

\begin{lstlisting}[label=code:checksumsold, caption={Хеш-суммы, описанные в файле \texttt{checksums.txt}.}]
63 68 65 63 6b 73 75 6d  73 20 66 6f 72 6d 61 74  |checksums format|
20 76 65 72 73 69 6f 6e  3a 20 34 0a 67 cf 30 a7  | version: 4.g.0.|
25 ce b0 4e 0e 75 28 87  2a 91 be 42 82 8e 00 00  |%..N.u(.*..B....|
00 84 00 00 00 f1 3c 04  09 63 6f 75 6e 74 2e 74  |......<..count.t|
78 74 01 96 fc d9 4b 53  ab 22 6f 33 9a be c4 ce  |xt....KS."o3....|
28 da 5d 00 08 64 61 74  61 2e 62 69 6e 8b 01 5c  |(.]..data.bin..\|
70 1c a6 71 6a e6 e7 f3  d7 70 27 28 d3 99 12 01  |p..qj....p'(....|
40 24 e8 86 e7 dd 22 a9  de f5 3b 07 c6 0f 91 6d  |@$...."...;....m|
7e 09 2d 00 f0 25 6d 72  6b 33 70 eb db 30 3d 91  |~.-..%mrk3p..0=.|
74 86 b7 33 12 22 5a 25  89 00 cb 00 0b 70 72 69  |t..3."Z%.....pri|
6d 61 72 79 2e 69 64 78  10 f6 15 da b1 a5 50 1e  |mary.idx......P.|
18 54 ae 68 d7 22 30 aa  ee 00                    |.T.h."0...|
\end{lstlisting}

Хеш-суммы подсчитываются для файла с данными, файла с индексами, файла с количеством элементов в куске и файла с засечками.

%\cite{https://clickhouse.com/codebrowser/ClickHouse/src/Storages/MergeTree/MergeTreeDataPartWriterCompact.cpp.html#_ZN2DB30MergeTreeDataPartWriterCompact14addToChecksumsERNS_26MergeTreeDataPartChecksumsE}

% (https://clickhouse.com/codebrowser/ClickHouse/src/Storages/MergeTree/MergeTreeDataPartChecksum.cpp.html#_ZNK2DB26MergeTreeDataPartChecksums5writeERNS_11WriteBufferE)
Данные хранятся в сжатом виде, жмутся кодеком по умолчанию (LZ4). Порядок записи следующий:
\begin{itemize}
    \item [---] строка <<checksums format version: 4>>;
    \item [---] хеш-сумма сжимающего буфера (16 байт);
    \item [---] заголовок кодека (9 байт);
    \item [---] количество файлов (как число переменной длины, в данном случае 1);
    \item [---] для каждого файла:
        \begin{itemize}
            \item [---] имя файла;
            \item [---] размер файла;
            \item [---] хеш-сумма файла;
            \item [---] булевый признак сжатия;
            \item [---] размер расжатых данных (если булевый признак сжатия --- истина);
            \item [---] хеш-сумма расжатых данных (если булевый признак сжатия --- истина).
        \end{itemize}
\end{itemize}

С учетом того, что менялся только \texttt{data.bin}, требуется поменять хеш-сумму только для него. Содержимое файла после изменений представлено в листинге \ref{code:checksumsnew}. Хеш-сумма не была сжата, поэтому размер файла не изменился.

\begin{lstlisting}[label=code:checksumsnew, caption={Хеш-суммы, описанные в файле \texttt{checksums.txt} после внесения изменений.}]
63 68 65 63 6b 73 75 6d  73 20 66 6f 72 6d 61 74  |checksums format|
20 76 65 72 73 69 6f 6e  3a 20 34 0a 67 cf 30 a7  | version: 4.g.0.|
25 ce b0 4e 0e 75 28 87  2a 91 be 42 82 8e 00 00  |%..N.u(.*..B....|
00 84 00 00 00 f1 3c 04  09 63 6f 75 6e 74 2e 74  |......<..count.t|
78 74 01 96 fc d9 4b 53  ab 22 6f 33 9a be c4 ce  |xt....KS."o3....|
28 da 5d 00 08 64 61 74  61 2e 62 69 6e 8b 01 5c  |(.]..data.bin..\|
70 1c a6 71 6a e6 e7 f3  d7 70 27 28 d3 99 12 01  |p..qj....p'(....|
40 24 e8 86 e7 dd 22 a9  de f5 3b 07 c6 0f 91 6d  |@$...."...;....m|
7e 09 2d 00 f0 25 6d 72  6b 33 70 eb db 30 3d 91  |~.-..%mrk3p..0=.|
74 86 b7 33 12 22 5a 25  89 00 cb 00 0b 70 72 69  |t..3."Z%.....pri|
6d 61 72 79 2e 69 64 78  10 f6 15 da b1 a5 50 1e  |mary.idx......P.|
18 54 ae 68 d7 22 30 aa  ee 00                    |.T.h."0...|
\end{lstlisting}

Вывод проверки после проведенных изменений показан в листинге \ref{code:errchain}.

\begin{lstlisting}[label=code:errchain, caption={Результат проверки после исправления \texttt{checksums.txt}}]
CHECK TABLE cats
FORMAT Vertical

Query id: dbcbc244-10f0-491d-85e7-849377ac73e0

Row 1:
──────
part_path: all_1_1_0
is_passed: 1
message:

Row 2:
──────
part_path: parts chain
is_passed: 0
message:   One or more parts before all_1_1_0 were deleted OR part itself is not valid!
\end{lstlisting}

Проверка показывает, что теперь нарушена цепь хеш-сумм блоков. Полный алгоритм построения цепи хеш-сумм и код алгоритма описаны в конструкторском и технологическом разделах. Файл с цепью хеш-сумм содержит в себе информацию о массиве чисел размером 8 байт. для 1 куска содержится 16 байт информации. После пересчета хеш-суммы и замены значений в файле \\ \texttt{commited\_parts\_chain.bin} вывод не поменяется, так как данные о значениях цепочки кэшируются в оперативной памяти. Данный факт вынуждает злоумышленника сделать так, чтобы СУБД перезапустилась. При перезапуске сервера значения цепи хеш-сумм поменяются на нужные и получится вывод, показанный в листинге \ref{code:checkok}.

\begin{lstlisting}[label=code:checkok, caption={Результат проверки после исправления \texttt{commited\_parts\_chain.bin}.}]
CHECK TABLE cats
FORMAT Vertical

Query id: 95e1ae1c-203e-488b-a0a2-f3885ab4aff1

Row 1:
──────
part_path: all_1_1_0
is_passed: 1
message:

Row 2:
──────
part_path: parts chain
is_passed: 1
message:
\end{lstlisting}

Как итог, чтобы проверка куска проходила успешно, требуется:
\begin{enumerate}
    \item пересчитать значение хеш-суммы сжатого блока, в который было внесено изменение;
    \item пересчитать значение хеш-суммы файла \texttt{data.bin} в файле \texttt{checksums.txt};
    \item пересчитать значение цепи хеш-сумм в файле \texttt{commited\_parts\_chain.bin}.
\end{enumerate}

При удалении потребуется сделать только последний шаг. Аналогично потребуется поступить при замене куска (то есть при удалении и вставке куска с аналогичной структурой). Подобные сценарии неправомерного доступа проходили бы при старом методе хранения данных.

\subsubsection{Широкое хранение}

Для широкого хранения следует рассмотреть отличия от приведенного выше алгоритма действий:
\begin{enumerate}
    \item изменение сжатого блока и его хеш-суммы будет происходит в соответсвующем нужному столбцу файле;
    \item изменение в файле \texttt{checksums.txt} будет происходить для файла, соответствующего меняемому столбцу.
\end{enumerate}

\subsection{Работа системы с шифрованием данных}

Шифрование можно настроить на нескольких уровнях:
\begin{itemize}
    \item [---] на уровне отдельных столбцов;
    \item [---] на уровне виртуального диска (пути в файловой системе, который будет передан ClickHouse через конфигурационный файл).
\end{itemize}

Шифрование на уровне столбцов позволяет злоумышленнику менять файлы с индексами, а также файлы с засечками, что в совокупности позволяет сделать так, чтобы при выборке данных с условием на первичный ключ данные какого-либо куска не читались вовсе, что аналогично их удалению. Однако на запросы, которые не имеют ограничений на первичный ключ, данный подход не подействует.

Пусть имеется шифрование на уровне виртуального диска. Пример конфигурации для такого шифрования приведен в листинге \ref{code:config}.

\begin{lstlisting}[label=code:config, caption={Параметры конфигурации для шифрования на уровне диска.}]
<storage_configuration>
    <disks>
        <disk1>
            <type>local</type>
            <path>/ClickHouse/build/programs/disk/</path>
        </disk1>
        <default>
            <type>encrypted</type>
            <disk>disk1</disk>
            <path>enc/</path>
            <algorithm>AES_256_CTR</algorithm>
            <key>12340000000000001234000000000000</key>
        </default>
    </disks>
</storage_configuration>
\end{lstlisting}

Тогда в случае повторения запросов из листинга \ref{code:createtable1}, данные, записанные в файл, будут выглядеть так, как показано в листинге \ref{code:encrypted}.

\begin{lstlisting}[label=code:encrypted, caption={Данные при шифровании диска.}]
45 4e 43 01 00 02 00 00  00 00 00 00 00 00 00 05  |ENC.............|
31 8d 4f 11 c1 c1 3b 78  ae 20 b6 59 7b ff 67 f4  |1.O...;x. .Y{.g.|
00 00 00 00 00 00 00 00  00 00 00 00 00 00 00 00  |................|
f9 82 0a 75 5f 80 01 de  36 e4 22 e9 78 b1 5a 8e  |...u_...6.".x.Z.|
8a 24 e2 2b 2b 57 5b cc  71 18 96 ea af ce 21 e8  |.$.++W[.q.....!.|
e7 20 5c 8b 26 20 98 8d  e9 34 17 cb a9 82 72 43  |. \.& ...4....rC|
ba fc b6 3b f6 f2 78 a2  89 a0 a2 80 b5 4e e5 3b  |...;..x......N.;|
b3 30 f9 e5 6b 20 f5 c0  ef 77 a2 6b 2f ce 4e 0c  |.0..k ...w.k/.N.|
e5 31 af 8c 1c 2f 55 e1  27 9d db 70 24 bc 92 de  |.1.../U.'..p$...|
3e 8c 91 12 83 1c 90 1a  f7 2b a1 1a ce c4 56 c3  |>........+....V.|
a7 e0 62 f8 92 d7 c1 e5  23 eb 64 df 79 5f 15 3d  |..b.....#.d.y_.=|
02 b4 c0 e6 de 12 f0 b2  a4 f9 7e                 |..........~|
\end{lstlisting}

Шифрование не позволяет исправить хеш-суммы подобно тому, как это выполнялось без него. Требуется проверить:
\begin{itemize}
    \item [---] что работают все стандартные операции:
        \begin{itemize}
            \item [---] вставка;
            \item [---] слияние;
            \item [---] мутация.
        \end{itemize}
    \item [---] что неправомерные удаление и изменение блока обнаружаются.
\end{itemize}

\pagebreak

\subsubsection{Проверка работы стандартных операций}

Для проверки корректности работы цепи хеш-сумм со вставкой можно сделать 3 вставки командой из листинга \ref{code:createtable1}. Результат проверки таблицы в листинге \ref{code:check1}.

\begin{lstlisting}[label=code:check1, caption={Результат проверки после вставки 4 блоков.}]
CHECK TABLE cats
FORMAT Vertical

Query id: 70296b3b-a0ec-4938-b731-a47310bc32b8

Row 1:
──────
part_path: all_1_1_0
is_passed: 1
message:

Row 2:
──────
part_path: all_2_2_0
is_passed: 1
message:

Row 3:
──────
part_path: all_3_3_0
is_passed: 1
message:

Row 4:
──────
part_path: all_4_4_0
is_passed: 1
message:

Row 5:
──────
part_path: parts chain
is_passed: 1
message:
\end{lstlisting}

Для проверки слияния можно сделать еще 3 вставки. Результат проверки в листинге \ref{code:check2}. Из проверки видно, что произошло слияние блоков с 1 по 5, цепь осталась валидной.

\begin{lstlisting}[label=code:check2, caption={Результат проверки после вставки 7 блоков.}]
CHECK TABLE cats
FORMAT Vertical

Query id: c0888295-5a25-4528-8a96-7e8ba7762d00

Row 1:
──────
part_path: all_1_5_1
is_passed: 1
message:

Row 2:
──────
part_path: all_6_6_0
is_passed: 1
message:

Row 3:
──────
part_path: all_7_7_0
is_passed: 1
message:

Row 4:
──────
part_path: parts chain
is_passed: 1
message:
\end{lstlisting}

Для проверки работы мутаций следует выполнить запрос из листинга \ref{code:mut}. Результат проверки таблицы после проведения мутации представлен в листинге \ref{code:check3}, цепь хеш-сумм осталась валидной.

\begin{lstlisting}[language=sql, label=code:mut, caption={Запрос мутации.}]
ALTER TABLE cats UPDATE age=age + 1 WHERE age < 17;
\end{lstlisting}

\pagebreak

\begin{lstlisting}[label=code:check3, caption={Проверка таблицы после мутации.}]
CHECK TABLE cats
FORMAT Vertical

Query id: aed93f24-7417-4424-8d16-cac981c1d240

Row 1:
──────
part_path: all_1_5_1_8
is_passed: 1
message:

Row 2:
──────
part_path: all_6_6_0_8
is_passed: 1
message:

Row 3:
──────
part_path: all_7_7_0_8
is_passed: 1
message:

Row 4:
──────
part_path: parts chain
is_passed: 1
message:
\end{lstlisting}

\subsubsection{Проверка обнаружения неправомерных удаления и изменения кусков}

Для проверки обнаружения неправомерного удаления можно удалить кусок \texttt{all\_7\_7\_0\_8}. Вывод проверки представлен в листинге \ref{code:checkerr1}, вывод проверки после перезагрузки системы представлен в листинге \ref{code:checkerr2}. В обоих случаях показано, что данные были изменены где-то вне системы.

\pagebreak

\begin{lstlisting}[label=code:checkerr1, caption={Вывод системы после неправомерного удаления куска.}]
CHECK TABLE cats
FORMAT Vertical

Query id: 6aa87471-30e7-4c8c-8c79-e3243ee936ae

Row 1:
──────
part_path: all_1_5_1_8
is_passed: 1
message:

Row 2:
──────
part_path: all_6_6_0_8
is_passed: 1
message:

Row 3:
──────
part_path: all_7_7_0_8
is_passed: 0
message:   Check of part finished with error: 'Cannot open file /ClickHouse/build/programs/disk/enc/store/cf0/cf01ce53-cba4-4afb-8e11-26b487355f84/all_7_7_0_8/columns.txt, errno: 2, strerror: No such file or directory'
\end{lstlisting}

\pagebreak

\begin{lstlisting}[label=code:checkerr2, caption={Вывод системы после неправомерного удаления куска после перезагрузки системы.}]
CHECK TABLE cats
FORMAT Vertical

Query id: 362f64b5-a480-4f2a-b1ad-0c021a110c53

Row 1:
──────
part_path: all_1_5_1_8
is_passed: 1
message:

Row 2:
──────
part_path: all_6_6_0_8
is_passed: 1
message:

Row 3:
──────
part_path: parts chain
is_passed: 0
message:   Some parts in the end of the chain were deleted
\end{lstlisting}

Для проверки обнаружения неправомерного изменения можно изменить последовательность байт в зашифрованном файле с данными одного из кусков, например, \texttt{all\_7\_7\_0\_8}. После изменения данных командой из листинга \ref{code:dd} получается результат проверки из листинга \ref{code:checkattack}. Данное было бы обнаружено и до оптимизации метода, потому что, как было замечено в констуркторской части, при изменении данных изменяются еще и хеш-суммы файлов и сжатых блоков, которые были и до модификации метода.

\begin{lstlisting}[label=code:dd, caption={Неправомерное изменение данных.}]
printf '\x31' | dd of=cats/all_7_7_0_8/data.bin bs=1 seek=183 count=1 conv=notrunc
\end{lstlisting}

\pagebreak

\begin{lstlisting}[label=code:checkattack, caption={Результат проверки после неправомерного изменения данных.}]
CHECK TABLE cats
FORMAT Vertical

Query id: 3b86fbe1-b13e-4006-8e92-0d460a52784f

Row 1:
──────
part_path: all_1_5_1_8
is_passed: 1
message:

Row 2:
──────
part_path: all_6_6_0_8
is_passed: 1
message:

Row 3:
──────
part_path: all_7_7_0_8
is_passed: 0
message:   Checksum mismatch for file data.bin in data part
\end{lstlisting}

\subsection*{Вывод}

В данном разделе были рассмотрены способы обхода защиты системы, работающей без шифрования, было показано правильное функционирование с шифрованием и применением стандартных операций, а также проверена возможность обнаружения неправомерных действий.

\pagebreak