\section*{ЗАКЛЮЧЕНИЕ}
\addcontentsline{toc}{section}{ЗАКЛЮЧЕНИЕ}

В рамках данной работы:
\begin{itemize}
	\item [---] были классифицированы виды защиты информации при блочном хранении данных;
	\item [---] были рассмотрены основные понятия и структуры, используемые при проектировании методов хранения информации с возможностью защиты от неправомерного доступа;
	\item [---] был проведен анализ существующих методов блочного хранения данных с защитой от неправомерного доступа;
	\item [---] были выявлены зависимости между свойствами системы и видами защиты информации при блочном хранении данных;
	\item [---] была рассмотрена архитектура СУБД ClickHouse и структура хранения данных в движке MergeTree;
	\item [---] был проведен анализ блочного хранения данных в СУБД ClickHouse в движке MergeTree на предмет возможности защиты от неправомерного доступа, в частности --- возможности доказательства неправомерного доступа;
	\item [---] был спроектирован новый метод блочного хранения данных на основе имеющегося, предоставляющий возможность доказательства неправомерного доступа;
	\item [---] спроектированный метод был реализован, протестирован на предмет обхода защиты при различных конфигурациях системы, проверен на правильность при наличии шифрования в системе.
\end{itemize}

Таким образом цель работы --- разработать метод хранения данных с возможностью доказательства неправомерного доступа на основе хеш-сумм была достигнута.

\pagebreak