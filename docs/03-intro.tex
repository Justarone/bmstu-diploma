\section*{ВВЕДЕНИЕ}
\addcontentsline{toc}{section}{ВВЕДЕНИЕ}

Обеспечение защиты от неправомерного доступа --- мера по защите информации. При работе с информацией лицо, имеющее доступ к этой информации, ограничено правами, которые определяют способы взаимодействия с информацией. В качестве примера таких прав можно привести права на доступ, копирование, уничтожение, изменение, блокирование, распространения и предоставления информации.

Задача по обеспечению защиты от неправомерного доступа может решаться на нескольких уровнях:
\begin{itemize}
    \item[---] Непосредственная защита от неправомерного доступа. Самый желаемый уровень, при котором лицо не имеет возможности превышения прав при работе с информацией. Для чтения, например, может достигаться за счет шифрования данных на диске, где ключ шифрования имеется только у людей, обладающими правами на чтение.
    \item[---] Возможность устранения последствий неправомерного доступа. Данный уровень может быть использован совместно с уровнем выше, обеспечивая дополнительную защиту. Примерами методов защиты, обеспечиваюими описанную возможность, могут послужить резервное копирование и репликация данных.
    \item[---] Возможность доказательства неправомерного доступа. Данный уровень служит для ответа на вопрос, являются ли данными целостностными. Для обеспечения такой возможности может использоваться, например, расчет контрольных сумм.
\end{itemize}

Цель работы --- разработать метод хранения информации с возможностью доказательства неправомерного доступа на основе движка MergeTree\cite{MergeTree} в СУБД ClickHouse\cite{ch}.
 

Для достижения поставленной цели требуется решить следующие задачи:
\begin{itemize}
    \item[---] рассмотреть принципы работы существующих систем хранения данных;
    \item[---] рассмотреть общую архитектуру СУБД ClickHouse;
    \item[---] рассмотреть архитектуру движка MergeTree СУБД ClickHouse;
    \item[---] разработать метод хранения информации с возможностью доказательства несанкционированного доступа;
    \item[---] провести исследование метода на предмет способов и ограничений применения метода.
\end{itemize}

\pagebreak