\section*{ВВЕДЕНИЕ}
\addcontentsline{toc}{section}{ВВЕДЕНИЕ}

Информация ограниченного доступа --- информация, защита которой регулируется на законодательном уровне. Примерами такой информации могут послужить:
\begin{itemize}
    \item [---] коммерческая тайна;
    \item [---] персональные данные;
    \item [---] служебная тайна;
    \item [---] секрет производства.
\end{itemize}

При отсутствии обеспечения должного уровня защиты такой информации случаются ее утечки, что в свою очередь приводит к штрафам и другим серьезным последствиям. Начиная с 2006 года, число утечек информации растет, при этом наибольший и растущий канал утечки данных --- сеть \cite{infowatch}.

При работе с информацией лицо, имеющее доступ к этой информации, ограничено правами, которые определяют способы взаимодействия с информацией. В качестве примера таких прав можно привести права на чтение, изменение и удаление информации. Обеспечение защиты от неправомерного доступа --- мера по защите информации. Одним из видов такой защиты является возможность доказательства неправомерного доступа.

Цель работы --- разработать метод блочного хранения данных с возможностью доказательства неправомерного доступа на основе хеш-сумм.

Для достижения поставленной цели требуется решить следующие задачи:
\begin{itemize}
	\item [---] описать виды защиты информации, классифицировать их;
	\item [---] рассмотреть базовые элементы и понятия, используемые при проектировании методов хранения информации с возможностью защиты от неправомерного доступа;
	\item [---] провести анализ существующих методов хранения информации с защитой от неправомерного доступа;
	\item [---] провести анализ блочного хранения данных в системе, в которой планируется использование метода, на предмет защиты информации от неправомерного доступа;
	\item [---] спроектировать и реализовать метод блочного хранения данных с возмозможностью доказательства неправомерного доступа;
	\item [---] исследовать метод на предмет невозможности реализации угроз при различных конфигурациях системы.
\end{itemize}
\pagebreak