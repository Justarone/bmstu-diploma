\section*{СПИСОК ИСПОЛЬЗОВАННЫХ ИСТОЧНИКОВ}
\addcontentsline{toc}{section}{СПИСОК ИСПОЛЬЗОВАННЫХ ИСТОЧНИКОВ}

\begingroup
\renewcommand{\section}[2]{}
\begin{thebibliography}{}
    \bibitem{hash}
    Брюс Шнайер. <<Прикладная криптография. 2-е издание. Протоколы, алгоритмы и исходные тексты на языке С>>.
    \bibitem{hashsum}
    ГОСТ. Информационная технология. Криптографическая защита информации. Функция хеширования. [Электронный ресурс]. – Режим доступа: https://protect.gost.ru/v.aspx?control=8&id=172313 свободный – (13.10.2021).
    \bibitem{dbengine}
    Движки таблиц СУБД ClickHouse [Электронный ресурс]. – Режим доступа: https://clickhouse.com/docs/ru/engines/table-engines/ свободный – (23.09.2021).
    \bibitem{mysqlengines}
    Справочное руководство MySQL [Электронный ресурс]. – Режим доступа: https://dev.mysql.com/doc/refman/8.0/en/storage-engines.html свободный – (26.10.2021).
    \bibitem{crypto}
    Мао В. <<Современная криптография: Теория и практика>> — М.: Вильямс, 2005. — 768 с.
    \bibitem{dbms}
    Когаловский М.Р. <<Энциклопедия технологий баз данных.>> — М.: Финансы и статистика, 2002. — 800 с.
    \bibitem{bitcoin}
    Bitcoin: A Peer-to-Peer Electronic Cash System [Электронный ресурс]. – Режим доступа: https://bitcoin.org/bitcoin.pdf свободный – (15.09.2021).
    \bibitem{merkle}
    R.C. Merkle, "Protocols for public key cryptosystems," In Proc. 1980 Symposium on Security and Privacy, IEEE Computer Society, pages 122-133, April 1980.
    \bibitem{merkledag}
    Merkle Directed Acyclic Graphs (DAGs) [Электронный ресурс]. – Режим доступа: https://docs.ipfs.io/concepts/merkle-dag/ свободный – (17.09.2021).
    \bibitem{graphs}
    Дистель, Рейнхард (2005), ''Graph Theory (3rd ed.)'', Berlin, New York: Springer-Verlag.
    \bibitem{topsort}
    Левитин А. В. Глава 5. Метод уменьшения размера задачи: Топологическая сортировка // Алгоритмы. Введение в разработку и анализ — М.: Вильямс, 2006. — С. 220—224. — 576 с.
    \bibitem{git}
    Git Internals --- Git Objects [Электронный ресурс]. – Режим доступа: https://git-scm.com/book/en/v2/Git-Internals-Git-Objects свободный – (28.09.2021).
    \bibitem{pasis}
    G. Goodson, J. Wylie, G. Ganger, and M. Reiter, ''Efficient Byzantine-tolerant Erasure-coded Storage,'' Proceedings of the International Conference on Dependable Systems and Networks (DSN-2004). Florence, Italy, 2004. Supercedes Carnegie Mellon University Parallel Data Lab Technical Report CMU-PDL-03-104, December 2003
    \bibitem{erasurecode}
    Dave K. Kythe, Prem K. Kythe. Algebraic and Stochastic Coding Theory. — 1-е изд. — CRC Press, 2012. — С. 375—395. — 512 с.
    \bibitem{tcfs}
    G.Cattaneo, L. Catuogno, A. Del Sorbo, and P. Persiano, ''A Transparent Cryptographic File System for Unix,'' In Proceedings of the USENIX Annual Technical Conference, Freenix Track. Boston, MA, 2001.
    \bibitem{ncryptfs}
    C. Wright, M. Martino, and E. Zadok. “NcryptFS: A Secure and Convenient Cryptographic File System,” In Proceedings of the USENIX Conference General Track, San Antonio, TX, June 2003.
    \bibitem{linux}
    Уорд Б. Внутреннее устройство Linux. — СПб.: Питер, 2016. — 384 с.
    \bibitem{ocean}
    J. Kubiatowitz, D. Bindel, Y. Chen, S. Czerwinski, P. Eaton, D. Geels, R. Gummadi, S. Rhea, H. Weatherspoon, W. Weimer, C. Wells, and B. Zhao, ''Oceanstore: An Architecture for Global-Scale Persistent Storage,'' In Proceedings of the Ninth International Conference on Architectural Support for Programming Languages and Operating Systems, Cambridge, MA, Nov 2000.
    \bibitem{selfverify}
    H. Weatherspoon, C. Wells, and J. Kubiatowicz, “Naming and Integrity: Self-Verifying Data in Peer-to-Peer Systems,” In Proceedings of the International Workshop on Future Directions in Distributed Computing (FuDiCo 2002), June 2002.
    \bibitem{pki}
    Полянская О. Ю., Горбатов В. С. Инфраструктуры открытых ключей. Учебное пособие., Москва, 2007.
    \bibitem{ch}
    СУБД ClickHouse [Электронный ресурс]. – Режим доступа: https://clickhouse.com/ свободный – (03.09.2021).
    \bibitem{MergeTree}
    MergeTree движок СУБД ClickHouse [Электронный ресурс]. – Режим доступа: https://clickhouse.com/docs/ru/engines/table-engines/mergetree-family/mergetree/ свободный – (05.09.2021).
    \bibitem{mergetreearch}
    Архитектура СУБД ClickHouse. Движок MergeTree. [Электронный ресурс]. – Режим доступа: https://clickhouse.com/docs/ru/engines/table-engines/mergetree-family/mergetree/ свободный – (05.09.2021).
    \bibitem{datapartcompact}
    Компактный вид хранения данных куска. [Электронный ресурс]. – Режим доступа: https://github.com/ClickHouse/ClickHouse/blob/009e71e273238e968be5f5d3d144d838d0162e86/src/Storages/MergeTree/MergeTreeDataPartCompact.h#L18 свободный – (23.02.2022).
    \bibitem{datapartwide}
    Широкий вид хранения данных куска. [Электронный ресурс]. – Режим доступа: https://github.com/ClickHouse/ClickHouse/blob/009e71e273238e968be5f5d3d144d838d0162e86/src/Storages/MergeTree/MergeTreeDataPartWide.h#L15 свободный – (23.02.2022).
    \bibitem{hashingbuf}
    Хеширующий буфер. [Электронный ресурс]. – Режим доступа: https://github.com/ClickHouse/ClickHouse/blob/009e71e273238e968be5f5d3d144d838d0162e86/src/IO/HashingWriteBuffer.h#L15 свободный – (23.02.2022).
    \bibitem{compressingbuf}
    Сжимающий буфер. [Электронный ресурс]. – Режим доступа: https://github.com/ClickHouse/ClickHouse/blob/009e71e273238e968be5f5d3d144d838d0162e86/src/Compression/CompressedWriteBuffer.h#L16 свободный – (23.02.2022).
    \bibitem{filebuf}
    Файловый буфер. [Электронный ресурс]. – Режим доступа: https://github.com/ClickHouse/ClickHouse/blob/009e71e273238e968be5f5d3d144d838d0162e86/src/IO/WriteBufferFromFileDescriptor.h#L11 свободный – (23.02.2022).
    \bibitem{inserttemp}
    Запись временного куска при вставке. [Электронный ресурс]. – Режим доступа: https://github.com/ClickHouse/ClickHouse/blob/009e71e273238e968be5f5d3d144d838d0162e86/src/Storages/MergeTree/MergeTreeSink.cpp#L61 свободный – (23.02.2022).
    \bibitem{insertrename}
    Переименование куска при вставке. [Электронный ресурс]. – Режим доступа: https://github.com/ClickHouse/ClickHouse/blob/009e71e273238e968be5f5d3d144d838d0162e86/src/Storages/MergeTree/MergeTreeSink.cpp#L107 свободный – (23.02.2022).
    \bibitem{backgroundproc}
    Метод планирования фоновых задач. [Электронный ресурс]. – Режим доступа: https://github.com/ClickHouse/ClickHouse/blob/009e71e273238e968be5f5d3d144d838d0162e86/src/Storages/StorageMergeTree.cpp#L1004 свободный – (23.02.2022).
    \bibitem{datapart}
    Структура представления куска. [Электронный ресурс]. – Режим доступа: https://github.com/ClickHouse/ClickHouse/blob/009e71e273238e968be5f5d3d144d838d0162e86/src/Storages/MergeTree/IMergeTreeDataPart.h#L44 свободный – (23.02.2022).
    \bibitem{covered}
    Покрываемые куски. [Электронный ресурс]. – Режим доступа: https://github.com/ClickHouse/ClickHouse/blob/009e71e273238e968be5f5d3d144d838d0162e86/src/Storages/MergeTree/MergeTreeData.cpp#L2566 свободный – (23.02.2022).
    \bibitem{nist}
    Transitioning the Use of Cryptographic Algorithms and Key Lengths [Электронный ресурс]. – Режим доступа: https://nvlpubs.nist.gov/nistpubs/SpecialPublications/NIST.SP.800-131Ar2.pdf свободный – (23.02.2022).
    \bibitem{journaldbms}
    К. Дж. Дейт <<Введение в системы баз данных>> — 8-е изд. — М.: «Вильямс», 2006. — С. 1328.
    \bibitem{atomicity}
    Amsterdam, Jonathan. <<Atomic File Transactions, Part 1>>. O'Reilly. 2016.
    \bibitem{cpp}
    Руководство по языку C++. [Электронный ресурс]. – Режим доступа: https://en.cppreference.com/w/ свободный – (21.03.2022).
    \bibitem{chcompiler}
    Выбор компилятора для сборки ClickHouse. [Электронный ресурс]. – Режим доступа: https://clickhouse.com/docs/en/development/developer-instruction/#c-compiler свободный – (24.03.2022).
    \bibitem{clang}
    Официальный сайт Clang. [Электронный ресурс]. – Режим доступа: https://clang.llvm.org/ свободный – (24.03.2022).
    \bibitem{cmake}
    Официальный сайт системы сборки CMake. [Электронный ресурс]. – Режим доступа: https://cmake.org/ свободный – (27.03.2022).
    \bibitem{ninjabuild}
    Официальный сайт системы сборки Ninja. [Электронный ресурс]. – Режим доступа: https://ninja-build.org/ свободный – (27.03.2022).
\end{thebibliography}
\endgroup

\pagebreak