\section*{СПИСОК ИСПОЛЬЗОВАННЫХ ИСТОЧНИКОВ}
\addcontentsline{toc}{section}{СПИСОК ИСПОЛЬЗОВАННЫХ ИСТОЧНИКОВ}

\begingroup
\renewcommand{\section}[2]{}
\begin{thebibliography}{}
    \bibitem{ch}
    СУБД ClickHouse [Электронный ресурс]. – Режим доступа: 
    https://clickhouse.com/
    свободный – (03.09.2021).
    \bibitem{MergeTree}
    MergeTree движок СУБД ClickHouse [Электронный ресурс]. – Режим доступа: 
    https://clickhouse.com/docs/ru/engines/table-engines/mergetree-family/mergetree/
    свободный – (05.09.2021).
    \bibitem{hash}
    Брюс Шнайер. <<Прикладная криптография. 2-е издание. Протоколы, алгоритмы и исходные тексты на языке С>>.
    \bibitem{crypto}
    Мао В. <<Современная криптография: Теория и практика>> — М.: Вильямс, 2005. — 768 с. — ISBN 5-8459-0847-7.
    \bibitem{bitcoin}
    Bitcoin: A Peer-to-Peer Electronic Cash System [Электронный ресурс]. – Режим доступа: 
    https://bitcoin.org/bitcoin.pdf
    свободный – (15.09.2021).
    \bibitem{merkle}
    R.C. Merkle, "Protocols for public key cryptosystems," In Proc. 1980 Symposium on Security and Privacy, IEEE Computer Society, pages 122-133, April 1980.
    \bibitem{merkledag}
    Merkle Directed Acyclic Graphs (DAGs) [Электронный ресурс]. – Режим доступа: 
    https://docs.ipfs.io/concepts/merkle-dag/
    свободный – (17.09.2021).
    \bibitem{graphs}
    Дистель, Рейнхард (2005), ''Graph Theory (3rd ed.)'', Berlin, New York: Springer-Verlag, ISBN 978-3-540-26183-4.
    \bibitem{topsort}
    Левитин А. В. Глава 5. Метод уменьшения размера задачи: Топологическая сортировка // Алгоритмы. Введение в разработку и анализ — М.: Вильямс, 2006. — С. 220—224. — 576 с. — ISBN 978-5-8459-0987-9
    \bibitem{git}
    Git Internals --- Git Objects [Электронный ресурс]. – Режим доступа: 
    https://git-scm.com/book/en/v2/Git-Internals-Git-Objects
    свободный – (28.09.2021).
    \bibitem{pasis}
    G. Goodson, J. Wylie, G. Ganger, and M. Reiter, ''Efficient Byzantine-tolerant Erasure-coded Storage,'' Proceedings of the International Conference on Dependable Systems and Networks (DSN-2004). Florence, Italy, 2004. Supercedes Carnegie Mellon University Parallel Data Lab Technical Report CMU-PDL-03-104, December 2003
    \bibitem{erasurecode}
    Dave K. Kythe, Prem K. Kythe. Algebraic and Stochastic Coding Theory. — 1-е изд. — CRC Press, 2012. — С. 375—395. — 512 с. — ISBN 978-1439881811.
    \bibitem{tcfs}
    G.Cattaneo, L. Catuogno, A. Del Sorbo, and P. Persiano, ''A Transparent Cryptographic File System for Unix,'' In Proceedings of the USENIX Annual Technical Conference, Freenix Track. Boston, MA, 2001.
    \bibitem{ncryptfs}
    C. Wright, M. Martino, and E. Zadok. “NcryptFS: A Secure and Convenient Cryptographic File System,” In Proceedings of the USENIX Conference General Track, San Antonio, TX, June 2003.
    \bibitem{linux}
    Уорд Б. Внутреннее устройство Linux. — СПб.: Питер, 2016. — 384 с. — ISBN 978-5-496-01952-1.
    \bibitem{ocean}
    J. Kubiatowitz, D. Bindel, Y. Chen, S. Czerwinski, P. Eaton, D. Geels, R. Gummadi, S. Rhea, H. Weatherspoon, W. Weimer, C. Wells, and B. Zhao, ''Oceanstore: An Architecture for Global-Scale Persistent Storage,'' In Proceedings of the Ninth International Conference on Architectural Support for Programming Languages and Operating Systems, Cambridge, MA, Nov 2000.
    \bibitem{selfverify}
    H. Weatherspoon, C. Wells, and J. Kubiatowicz, “Naming and Integrity: Self-Verifying Data in Peer-to-Peer Systems,” In Proceedings of the International Workshop on Future Directions in Distributed Computing (FuDiCo 2002), June 2002.
	\bibitem{pki}
	Полянская О. Ю., Горбатов В. С. Инфраструктуры открытых ключей. Учебное пособие., Москва, 2007. ISBN 978-5-94774-602-0
\end{thebibliography}
\endgroup

\pagebreak