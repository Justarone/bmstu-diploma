\section*{ВВЕДЕНИЕ}
\addcontentsline{toc}{section}{ВВЕДЕНИЕ}

Обеспечение защиты от неправомерного доступа --- мера по защите информации. При работе с информацией лицо, имеющее доступ к этой информации, ограничено правами, которые определяют способы взаимодействия с информацией. В качестве примера таких прав можно привести права на чтение, изменение и удаление информации. Возможность доказательства неправомерного доступа --- способность системы хранения данных реагировать на изменения в данных, происходящих не через эту систему, и сигнализировать об этом при запросе.

Цель работы --- реализовать улучшение метода блочного хранения данных в СУБД ClickHouse в движке MergeTree, дающее возможность доказательства неправомерного доступа.

Для достижения поставленной цели требуется решить следующие задачи:
\begin{itemize}
	\item [---] описать инструменты и технологии, требуемые для реализации нового метода;
	\item [---] реализовать новый метод в виде класса, отвечающего за управление цепью хеш-сумм, связывающую куски данных;
	\item [---] встроить полученную реализацию класса в функционал движка MergeTree;
	\item [---] исследовать метод на предмет возможности реализации угроз (обхода защиты) при нешифрованном хранении данных;
	\item [---] исследовать корректность работы метода при различных событиях системы при шифрованном хранении данных.
\end{itemize}
\pagebreak
